\chapter{The C Language}

% 
% 
% Copyright 2002 Jonathan Bartlett
% 
% Permission is granted to copy, distribute and/or modify this
% document under the terms of the GNU Free Documentation License,
% Version 1.1 or any later version published by the Free Software
% Foundation; with no Invariant Sections, with no Front-Cover Texts,
% and with no Back-Cover Texts.  A copy of the license is included in fdl.xml
% 

In this chapter we will begin to look at our first "real-world"
programming language.  Assembly language is the language used
at the machine's level, but most people (including me) find
coding in assembly language too cumbersome for normal use.
Many computer languages have been invented to make the programming
task easier.  Knowing a wide variety of languages is useful for
many reasons, including

\begin{itemize}\item Different languages are good for different types of projects 
\item Different companies have different standard languages, so knowing more languages makes your skills more marketable 
\item The more languages you know, the easier it is to pick up new ones 
\item Different languages are based on different concepts, which will help you to learn different and better ways of doing things 
\end{itemize}

This chapter focuses on the C language.

\section{Compiled and Interpretted Languages}

C is a \emph{compiled} language.  When you wrote in
assembly language, each instruction you wrote was translated into
exactly one machine instruction for processing.  With compilers,
an instruction can translate into one or hundreds of machine 
instructions.  In fact, depending on how advanced your compiler
is, it might even restructure parts of your code to make it faster.
In assembly language, what you write is what you get.

There are also languages that are \emph{translated}
languages.  These languages require that the user run a program
called a \emph{translator} (also called a 
\emph{run-time environment}) that in turn runs the
given program.  These are usually slower than compiled programs,
since the translator has to read and interpret the code as it goes 
along.  However, in well-made translators, this time can be
fairly negligible.  There is also a class of hybrid languages
which partially compile a program before execution into byte-codes,
which are only machine readable.  The translator can read the
byte-codes much faster than it can read the regular language, so
the reading step only happens once.

There are many reasons to choose one or the other.  Compiled programs
are nice, because you don't have to already have a translator installed
in the user's machine.  You have to have a compiler for the language,
but the users of your program don't.  In a translated language, you
have to be sure that the user has a translator for your program,
and that the computer knows which translator runs your program.

\section{Your First C Program}

As you may have noticed, I enjoy presenting you with a program first,
and then explaining how it works.  So, here is your first program,
which prints "Hello world" to the screen and exits.  Type it in, and
give it the name Hello-World.c

\begin{simpletyping}
\lstinputlisting{Hello-World-c}
\end{simpletyping}

As you can see, it's a pretty simple program.  To compile it,
run the command

\begin{simpletyping}
\begin{lstlisting}
gcc -o HelloWorld Hello-World.c
\end{lstlisting}
\end{simpletyping}

To run the program, do

\begin{simpletyping}
\begin{lstlisting}
./HelloWorld
\end{lstlisting}
\end{simpletyping}

Let's look at how this program was put together.

Comments in C are started with \icode{/*} and ended with
\icode{*/}.  Comments can span multiple lines, but 
many people prefer to start and end comments on the same line
so they don't get confused.

\icode{\#include <stdio.h>} is the first part of
the program.  This is a \emph{preprocessor directive}.
C compiling is split into two stages - the preprocessor and the
main compiler.  This directive tells the preprocessor to look
for the file \icodefilename{stdio.h} and paste it into
your program.  So, everything in \icodefilename{stdio.h} is
now in your program just as if you typed it there yourself.  The angle brackets
around the filename tell the compiler to look in its standard
paths for the file (\icodefilename{/usr/include} and 
\icodefilename{/usr/local/include}, usually).  If it was
in quotes, like \icode{\#include "stdio.h"} it would
look in the current directory for the file.  Anyway, 
\icodefilename{stdio.h} contains the declarations for 
the standard input and output functions and variables.   The next few
lines are simply comments about the program.

Then there is the line \icode{int main(int argc, char **argv)}.
This is the start of a function.  C Functions are declared with their 
name, arguments and return type.  This declaration says taht the functions
name is \icode{main}, it returns an \icode{int} 
(integer), and has two arguments - an \icode{int} called
\icode{argc} and a \icode{char **} called 
\icode{argv}.  You don't have to worry about where the arguments
are positioned on the stack - the C compiler takes care of that for you.
You also don't have to worry about loading values into and out of registers.
The \icode{main} function is a special function - it is the
start or all C programs.  It always takes two parameters.  The first 
parameter is the number of arguments given to this command, and the
second parameter is a list of the arguments that were given.

The next line is a function call.  In assembly language, you had to 
push the arguments of a function onto the stack, and then call the function.
C takes care of this complexity for you.  You simply have to call the
function with the parameters in parenthesis.  In this case, we call
the function \icode{puts}, with a single parameter.  This
parameter is the 

% they do in assembly language, except that 

% 
% 
% Advanced Linking Techniques
% 
% LD\_PRELOAD
% 
% loading a library on the fly
% 
% static libraries
% 
% sonames
% 
% ELF sections and special sections (.ini .fini or whatever)
% 
% Process for loading and linking ELF files
% 
% Other object file types (COFF a.out)
% 
% The file format used to store programs in Linux is called ELF, which
% stands for "Executable and Linking Format".  ELF is very flexible,
% but it is also fairly complex.  Let's take a look at how we've
% been using ELF so far.
% 

% 
% <sect2>
% \textbf{ELF Sections}
% 

% The \icode{.section} directives we've been using
% are commands that tell the linker, \icode{ld}, to
% group things together.  There are many predefined sections.
% The ones we mainly use are \icode{.data} and 
% \icode{.text}.  In your assembly code, you can mix
% \icode{.data} and \icode{.text} sections,
% and the linker will put everything in a section together.  Let's
% take a look at one of our files.  Let's look at the 
% \icode{toupper} file we created earlier.  If you
% run the command \icode{objdump -h toupper}, you will
% get a listing that looks like this:
% 

% 
% 

\section{Review}

\section{Know the Concepts}

\begin{itemize}\item  
\end{itemize}

\section{Use the Concepts}

\begin{itemize}\item  
\end{itemize}

\section{Going Further}

\begin{itemize}\item  
\end{itemize}

